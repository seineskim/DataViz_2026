\documentclass[12pt,letterpaper]{article}
\usepackage{graphicx,textcomp}
\usepackage{natbib}
\usepackage{setspace}
\usepackage{fullpage}
\usepackage{color}
\usepackage[reqno]{amsmath}
\usepackage{amsthm}
\usepackage{fancyvrb}
\usepackage{amssymb,enumerate}
\usepackage[all]{xy}
\usepackage{endnotes}
\usepackage{lscape}
\newtheorem{com}{Comment}
\usepackage{float}
\usepackage{hyperref}
\newtheorem{lem} {Lemma}
\newtheorem{prop}{Proposition}
\newtheorem{thm}{Theorem}
\newtheorem{defn}{Definition}
\newtheorem{cor}{Corollary}
\newtheorem{obs}{Observation}
\usepackage[compact]{titlesec}
\usepackage{dcolumn}
\usepackage{tikz}
\usetikzlibrary{arrows}
\usepackage{multirow}
\usepackage{xcolor}
\newcolumntype{.}{D{.}{.}{-1}}
\newcolumntype{d}[1]{D{.}{.}{#1}}
\definecolor{light-gray}{gray}{0.65}
\usepackage{url}
\usepackage{listings}
\usepackage{color}

\definecolor{codegreen}{rgb}{0,0.6,0}
\definecolor{codegray}{rgb}{0.5,0.5,0.5}
\definecolor{codepurple}{rgb}{0.58,0,0.82}
\definecolor{backcolour}{rgb}{0.95,0.95,0.92}

\lstdefinestyle{mystyle}{
	backgroundcolor=\color{backcolour},   
	commentstyle=\color{codegreen},
	keywordstyle=\color{magenta},
	numberstyle=\tiny\color{codegray},
	stringstyle=\color{codepurple},
	basicstyle=\footnotesize,
	breakatwhitespace=false,         
	breaklines=true,                 
	captionpos=b,                    
	keepspaces=true,                 
	numbers=left,                    
	numbersep=5pt,                  
	showspaces=false,                
	showstringspaces=false,
	showtabs=false,                  
	tabsize=2
}
\lstset{style=mystyle}
\newcommand{\Sref}[1]{Section~\ref{#1}}
\newtheorem{hyp}{Hypothesis}

\title{Problem Set 3}
\date{Due: February 18, 2026}
\author{Data Visualisation for Social Scientists}

\begin{document}
	\maketitle
	
	\section*{Instructions}
	\begin{itemize}
	\item Please show your work! You may lose points by simply writing in the answer. If the problem requires you to execute commands in \texttt{R}, please include the code you used to get your answers. Please also include the \texttt{.R} file that contains your code. If you are not sure if work needs to be shown for a particular problem, please ask.
\item Your homework should be submitted electronically on GitHub.
\item This problem set is due before 23:59 on Wednesday February 18, 2026. No late assignments will be accepted.
	\end{itemize}
	
	\vspace{.25cm}
	\section*{Canadian Election Study}
	
The data for this problem set come from the	Canadian Election Study (\href{https://ces-eec.sites.olt.ubc.ca/files/2017/04/CES2015_Combined_Data_Codebook.pdf}{CES})  in 2015. The main purpose of the study is to give a comprehensive picture of the Canadian election: why people vote as they do, what changes during campaigns and across elections, and how Canadian voting compares with that in other democracies.

\subsection*{Data Manipulation}

\begin{enumerate}
\item Load the CES .csv file from \href{https://raw.githubusercontent.com/ASDS-TCD/DataViz_2026/refs/heads/main/datasets/CES2015.csv}{GitHub} into your global environment. Filter respondents to only include "high quality" participants: 
\begin{verbatim}
ces2015 <- ces2015 |> filter(discard == "Good quality")
\end{verbatim}

\lstinputlisting[
language=R,
firstline=17,
lastline=21
]{PS03_SK.R}


\item Filter the dataset to those participants that answered the question about voting for the past election using \texttt{p\_voted}. Consider respondents who gave a "Yes" answer as having voted, while “No” as not having voted. Treat “Don’t know” and “Refused” as missing.

\lstinputlisting[
language=R,
firstline=25,
lastline=31
]{PS03_SK.R}

\item Create an age variable and group into categories (e.g., $<$30, 30-44, 45-64, 65+). Year of birth is in age (four‑digit year).
	\lstinputlisting[
language=R,
firstline=35,
lastline=44
]{PS03_SK.R}
\end{enumerate}

\subsection*{Data Visualization}

\begin{enumerate}
	\item Plot turnout rate by age group.
	
	\vspace{0.3cm}
	\textbf{Answer:} 
	I visualized the turnout rate by age group using a bar chart. A key step was converting the age categories into factors to ensure a logical chronological flow. This prevents R's default alphabetical sorting, which would mislead the reader. The Y-axis was formatted using \texttt{scales::percent} to clearly communicate the probability of voting in each cohort.
	\vspace{0.3cm}
	
		\lstinputlisting[
	language=R,
	firstline=53,
	lastline=81
	]{PS03_SK.R}
	
	\begin{center}
		\includegraphics[width=0.9\textwidth]{Plot1.pdf}
	\end{center}
	
	\item Create a density plot of ideology by party, restricting your sample to respondents with non‑missing left–right self‑placement (0–10 scale) and those that intended to vote for a main party (e.g., Liberal, Conservative, NDP, Bloc in Quebec, and Green).
	
	\vspace{0.3cm}
	\textbf{Answer:} 
	This density plot shows the ideological distribution across major parties. I first performed data cleaning by removing non-substantive outliers like '1000' from the \texttt{p\_selfplace} variable. I applied alpha transparency (0.4) to manage the overlapping distributions, making it easier to see where supporters of different parties cluster on the 0-10 left-right scale.
	\vspace{0.3cm}
	
	\lstinputlisting[
	language=R,
	firstline=89,
	lastline=125
	]{PS03_SK.R}
	
		\begin{center}
		\includegraphics[width=0.9\textwidth]{Plot2.pdf}
	\end{center}
	
	\item Produce histogram counts of turnout by income (\texttt{income\_full}), faceted by province.
	
	\vspace{0.3cm}
	\textbf{Answer:} 
	To analyze the relationship between income levels and turnout across different regions, I produced a histogram faceted by province. I utilized the \textbf{small multiples} technique via \texttt{facet\_wrap()} to allow for easy geographic comparisons without overcrowding a single plot. The \texttt{income\_full} variable was reordered into a logical progression from "less than \$29,999" to "more than \$110,000" to ensure the plot's \textbf{truthfulness} and intuitive readability.
	\vspace{0.3cm}
	
	\lstinputlisting[
	language=R,
	firstline=132,
	lastline=173
	]{PS03_SK.R}
	
		\begin{center}
		\includegraphics[width=0.9\textwidth]{Plot3.pdf}
	\end{center}
	
	\item Create your own reusable custom theme. Apply your theme to one of the previous plots and add:
	\begin{enumerate}
		\item An improved title summarizing the main substantive takeaway.
		
		\item A more informative subtitle describing the sample and variables.
		\item A caption noting data source, weighting, and key coding decisions.
		\item At least one direct annotation using \texttt{ggrepel} that calls out a key pattern.
		
		\vspace{0.3cm}
		\textbf{Answer:} 
		For the final requirement, I transformed the age-turnout analysis into a Lollipop chart to improve the data-to-ink ratio. By removing the bulk of the bars and focusing on the points, the visual noise is significantly reduced. I applied my custom theme, \texttt{my\_custom\_theme}, which follows CRAP principles by centering the title for alignment and using bold text for contrast. Finally, I used \texttt{ggrepel} for direct annotation to highlight the 65+ group's peak participation rate.
		\vspace{0.3cm}
		
		\lstinputlisting[
		language=R,
		firstline=181,
		lastline=239
		]{PS03_SK.R}
		
		\begin{center}
		\includegraphics[width=0.9\textwidth]{Final_Plot.pdf}
	\end{center}
		
	\end{enumerate}

\end{enumerate}


\end{document}
