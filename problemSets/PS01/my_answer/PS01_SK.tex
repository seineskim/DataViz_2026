\documentclass[12pt,letterpaper]{article}
\usepackage{graphicx,textcomp}
\usepackage{natbib}
\usepackage{setspace}
\usepackage{fullpage}
\usepackage{color}
\usepackage[reqno]{amsmath}
\usepackage{amsthm}
\usepackage{fancyvrb}
\usepackage{amssymb,enumerate}
\usepackage[all]{xy}
\usepackage{endnotes}
\usepackage{lscape}
\newtheorem{com}{Comment}
\usepackage{float}
\usepackage{hyperref}
\newtheorem{lem} {Lemma}
\newtheorem{prop}{Proposition}
\newtheorem{thm}{Theorem}
\newtheorem{defn}{Definition}
\newtheorem{cor}{Corollary}
\newtheorem{obs}{Observation}
\usepackage[compact]{titlesec}
\usepackage{dcolumn}
\usepackage{tikz}
\usetikzlibrary{arrows}
\usepackage{multirow}
\usepackage{xcolor}
\newcolumntype{.}{D{.}{.}{-1}}
\newcolumntype{d}[1]{D{.}{.}{#1}}
\definecolor{light-gray}{gray}{0.65}
\usepackage{url}
\usepackage{listings}
\usepackage{color}

\definecolor{codegreen}{rgb}{0,0.6,0}
\definecolor{codegray}{rgb}{0.5,0.5,0.5}
\definecolor{codepurple}{rgb}{0.58,0,0.82}
\definecolor{backcolour}{rgb}{0.95,0.95,0.92}

\lstdefinestyle{mystyle}{
	backgroundcolor=\color{backcolour},   
	commentstyle=\color{codegreen},
	keywordstyle=\color{magenta},
	numberstyle=\tiny\color{codegray},
	stringstyle=\color{codepurple},
	basicstyle=\footnotesize,
	breakatwhitespace=false,         
	breaklines=true,                 
	captionpos=b,                    
	keepspaces=true,                 
	numbers=left,                    
	numbersep=5pt,                  
	showspaces=false,                
	showstringspaces=false,
	showtabs=false,                  
	tabsize=2
}
\lstset{style=mystyle}
\newcommand{\Sref}[1]{Section~\ref{#1}}
\newtheorem{hyp}{Hypothesis}

\title{Problem Set 1}
\date{Due: January 28, 2026}
\author{Data Visualisation for Social Scientists}

\begin{document}
	\maketitle

	\section*{Roll Call Votes in the European Parliament}

\subsection*{Data Manipulation}

\begin{enumerate}
	
	%=================================================
	\item Load these datasets into your global environment:
	\begin{itemize}
		\item \texttt{mep\_info\_26Jul11.xls} (MEP characteristics, EP1--EP5)
		\item \texttt{rcv\_ep1.txt} (EP1 roll-call votes)
	\end{itemize}
	
	\noindent
	I load the EP1 roll-call vote data and the EP1 MEP information at the start of the workflow so that every step is reproducible from the original source files.
	
	\lstinputlisting[
	language=R,
	firstline=16,
	lastline=17
	]{PS01_SK.R}
	
	\lstinputlisting[
	language=R,
	firstline=19,
	lastline=21
	]{PS01_SK.R}
	
	%=================================================
	\item Briefly describe (2--3 sentences each) the unit of analysis and key variables in each of these two datasets.
	
	\textbf{MEP information dataset (\texttt{mep\_info\_26Jul11.xls}).}
	The unit of analysis is the individual Member of the European Parliament (MEP). Each row contains one MEP’s identifying information, party/group affiliation, and ideological coordinates. Key variables include \texttt{MEPID}, \texttt{MS}, \texttt{NP}, \texttt{EP Group}, and the NOMINATE dimensions \texttt{NOM-D1} and \texttt{NOM-D2}.
	
	\vspace{0.3cm}
	
	\textbf{Roll-call vote dataset (\texttt{rcv\_ep1.txt}).}
	In the original wide format, each row corresponds to one MEP and each roll-call vote is stored in separate columns (\texttt{V1}--\texttt{Vn}). After reshaping into long format, the unit of analysis becomes a single MEP’s vote on a single roll-call vote. Key variables include \texttt{MEPID}, \texttt{MEPNAME}, \texttt{MS}, \texttt{NP}, \texttt{EPG}, and the vote columns \texttt{V1}--\texttt{Vn}.
	
	\newpage
	%=================================================
	\item The \texttt{rcv\_ep1} data are in a wide format, with V1, V2, \dots, Vn as separate vote columns.
	\begin{itemize}
		\item Identify which columns are ID/metadata and which columns are vote decisions. Tidy the data so that each row is a single vote by a single MEP.
		\item Create a summary table of counts of decision categories across all votes.
	\end{itemize}
	
	\noindent
	I first separate ID/metadata variables (\texttt{MEPID, MEPNAME, MS, NP, EPG}) from the roll-call vote variables (\texttt{V1}--\texttt{Vn}). I then reshape the dataset from wide to long format using \texttt{pivot\_longer()} so that each row corresponds to one MEP--rollcall vote. Finally, I recode the numeric vote codes into substantive categories and compute category counts.
	
	\lstinputlisting[
	language=R,
	firstline=27,
	lastline=57
	]{PS01_SK.R}
	
	\medskip
	\noindent\textbf{Output (decision category counts).}
	\begin{Verbatim}[fontsize=\small]
decision                      n
<chr>                     <int>
1 Present but did not vote 109224
2 Not an MEP               103618
3 Absent                    99753
4 Yes                       88185
5 No                        75171
6 Abstain                    9577
	\end{Verbatim}
	
	%=================================================
	\item Construct a new dataset that combines MEP-level information with their vote decisions from EP1 in long format (from part 3). Check for missingness.
	
	\noindent
	I merge the long-format roll-call votes with the MEP-level information by \texttt{MEPID}. After merging, I summarize missingness in key variables to verify data quality (especially the NOMINATE dimensions and EP group).
	
	\lstinputlisting[
	language=R,
	firstline=62,
	lastline=73
	]{PS01_SK.R}
	
	\medskip
	\noindent\textbf{Output (missingness summary).}
	\begin{Verbatim}[fontsize=\small]
missing_nomd1 missing_nomd2 missing_epg
<int>         <int>       <int>
1           886           886         886
	\end{Verbatim}
	
	%=================================================
\item Compute, for each EP group in EP1:
\begin{itemize}
	\item The mean rate of Yes votes (Yes over Yes+No+Abstain) across all roll calls.
	\item The mean abstention rate.
	\item The mean vote preferences along NOM-D1 and NOM-D2.
\end{itemize}

\noindent
I restrict the data to valid voting decisions (Yes/No/Abstain; vote codes 1--3). 
For each EP group, I compute (i) the Yes rate as the mean of the indicator $\mathbb{1}(\text{vote}=Yes)$, 
(ii) the abstention rate as the mean of $\mathbb{1}(\text{vote}=Abstain)$, and (iii) the mean values of 
\texttt{NOM-D1} and \texttt{NOM-D2}. Because the NOMINATE variables contain ``.'' entries, I first recode them as missing and then convert them to numeric.

\lstinputlisting[
language=R,
firstline=81,
lastline=100
]{PS01_SK.R}

\medskip
\noindent\textbf{Results (from the EP-group summary table).}
\begin{enumerate}
	\item \textbf{Mean Yes rate.}
	The highest Yes rate is for EP Group N (0.581), followed by S (0.576) and M (0.528). 
	The lowest Yes rate is for EP Group C (0.415).
	
	\item \textbf{Mean abstention rate.}
	EP Group R has the highest abstention rate (0.265). All other groups have abstention rates below 0.10,
	with the lowest for EP Group E (0.0215).
	
	\item \textbf{Mean NOMINATE preferences (NOM-D1 and NOM-D2).}
	EP Group C has the highest mean NOM-D1 (0.811), whereas EP Group R has the lowest mean NOM-D1 (-0.586).
	For NOM-D2, EP Group C has the highest mean (0.531) and EP Group G has the lowest mean (-0.817).
\end{enumerate}

\medskip
\noindent\textbf{Full output (EP-group summary table).}
\begin{Verbatim}[fontsize=\small]
	`EP Group` yes_rate abstention_rate mean_nomdim1 mean_nomdim2
	<chr>         <dbl>           <dbl>        <dbl>        <dbl>
	1 C             0.415          0.0752       0.811        0.531 
	2 E             0.509          0.0215       0.513       -0.268 
	3 G             0.512          0.0697       0.289       -0.817 
	4 L             0.486          0.0632       0.420       -0.301 
	5 M             0.528          0.0800      -0.299       -0.149 
	6 N             0.581          0.0562       0.202       -0.195 
	7 R             0.457          0.265       -0.586       -0.0869
	8 S             0.576          0.0574      -0.0907       0.390 
\end{Verbatim}

\end{enumerate}
%=================================================
\newpage

\subsection*{Data Visualization}

\begin{enumerate}
	
	%=================================================
	\item Plot the distribution of the first NOMINATE dimension by EP group, and explain any trends you see.
	
	\begin{figure}[H]
		\centering
		\includegraphics[width=0.9\textwidth]{viz1_SK.pdf}
		\caption{Distribution of NOMINATE Dimension 1 by EP Group (EP1).}
	\end{figure}
	
	\noindent
	The boxplots show clear differences in the median and spread of NOMINATE Dimension 1 across EP groups. 
	For example, EP Group C is concentrated at higher NOM-D1 values, while EP Group R is centered on lower values, indicating ideological separation.
	
	\lstinputlisting[
	language=R,
	firstline=109,
	lastline=123
	]{PS01_SK.R}
	
	%=================================================
	\item Make a scatterplot of \textit{nomdim1} (x-axis) and \textit{nomdim2} (y-axis), with one point per MEP and color by EP group.
	
	\begin{figure}[H]
		\centering
		\includegraphics[width=0.9\textwidth]{viz2_SK.pdf}
		\caption{MEP ideal points on NOMINATE Dimensions 1 and 2, colored by EP Group.}
	\end{figure}
	
	\noindent
	MEPs cluster by EP group in the two-dimensional ideological space. While many groups occupy distinct regions, partial overlap remains, which may reflect ideological proximity between groups or within-group heterogeneity.
	
	\lstinputlisting[
	language=R,
	firstline=128,
	lastline=142
	]{PS01_SK.R}
	
	\newpage
	
	%=================================================
	\item Produce a boxplot of the proportion voting \textit{Yes} by EP group to visualize cohesion.
	
	\begin{figure}[H]
		\centering
		\includegraphics[width=0.9\textwidth]{viz3_SK.pdf}
		\caption{Distribution of MEP-level Yes-vote shares by EP Group (EP1).}
	\end{figure}
	
	\noindent
	Groups with narrower distributions of individual-level Yes shares exhibit higher internal cohesion, whereas wider distributions indicate greater variation in voting behavior within the group.
	
	\lstinputlisting[
	language=R,
	firstline=147,
	lastline=167
	]{PS01_SK.R}
	
	%=================================================
%=================================================
\item Display the proportion voting \textit{Yes} by national party using a bar plot.

\begin{figure}[H]
	\centering
	\includegraphics[width=0.95\textwidth]{viz4_SK.pdf}
	\caption{Yes-vote share by national party (EP1).}
\end{figure}

\noindent
Following the clarification provided after the assignment was released, I plot the proportion
of Yes votes by national party for the entire EP1 period (not by year).
For each national party, I compute the Yes share as:
\[
\text{Yes share} \;=\; \frac{N_{\text{Yes}}}{N_{\text{Yes}} + N_{\text{No}} + N_{\text{Abstain}}}.
\]
restricting to valid votes (codes 1--3). The bar plot shows substantial cross-party variation
in support for roll-call votes during EP1.

\lstinputlisting[
language=R,
firstline=172,
lastline=196
]{PS01_SK.R}

\end{enumerate}

\end{document}